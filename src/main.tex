%%
%% Author: Gerard Bosch (gerard.bosch@gmail.com)
%% 13/02/2018
%%

\documentclass[notitlepage, usenames,dvipsnames]{beamer}
\usefonttheme[onlymath]{serif}      %font serif (beamer usa sans-serif) per a l'entorn math
\usepackage{beamerprosper}
\usepackage{pstricks-add}
%\usepackage{beamertexpower}
\usepackage[utf8]{inputenc}
\usepackage[english]{babel}
%\usepackage{default}
\usepackage{verbatim}
\usepackage{multicol}
\usepackage{graphicx}
\usepackage{comment}

%\setbeameroption{show notes on second screen=right}
\setbeamertemplate{navigation symbols}{}

%% Aliases %%
\renewcommand{\tt}{\texttt}     %redefinit! és l'entorn typewriter, però sempre faig anar verbatims.
\newcommand{\ts}{\textsl}
\renewcommand{\ni}{\noindent}   %redefinit! és un simbol de math, \in al revés :S
\newcommand{\eh}{\emph}
\newcommand{\st}{\structure}


\usetheme[secheader]{Boadilla}
%\usetheme{Copenhagen}
%\usetheme{Darmstadt}
%\usetheme{Frankfurt}
%\usetheme{Goettingen}
%\usetheme{Hannover}
%\usetheme{Luebeck}
%\usetheme[secheader]{Madrid}
%\usetheme{Warsaw}


%\setbeamercolor{frametitle}{bg=gray!10}
%\setbeamercolor{block body alerted}{bg=red!20}
%\setbeamercolor{frametitle}{use=structure,bg=structure.fg!10!bg}
%\setbeamercolor{frametitle}{parent=normal text,use=block title,bg=block title.bg!50!bg}

% Custom colors
\definecolor{DarkBlue}{HTML}{0d0982}


%\hypersetup{pdfpagemode=FullScreen}


\title[Introduction to Blockchain Technology]{An introduction to Blockchain technology}
%\subtitle{}
\author[Gerard Bosch]{Gerard Bosch}
\institute{\email{gerard.bosch@gmail.com}}
\date{\today}

% \AtBeginSection[]{\frame[shrink]{\frametitle{Outline}\tableofcontents[currentsection]}} %posar [*] per aplicar-ho a \section*
\AtBeginSection[]{\frame{\frametitle{Outline}\begin{multicols}{2}
                                                 \tableofcontents[currentsection]
\end{multicols}}}    % 2 columns toc

% 2 columns table of contents !package multicol
% \begin{frame}
% \begin{multicols}{2}
%   \tableofcontents
% \end{multicols}
% \end{frame}

\setbeamercovered{transparent}


\begin{document}

    % TODO
    \begin{comment}
        Blockchain uses cryptographic mechanisms

        Crypto in crypto-currencies does not mean that all information in the blockchain is encrypted and secret\ldots

        Bitcoin blockchain is not confidential at all as transaction details are public
        \ldots but can be a difficult to trace back where the money came from.

        % Ref: https://en.bitcoin.it/wiki/How_bitcoin_works#Cryptography

        \st{Crypto} comes from the use of cryptographic techniques used by the operation of the protocol and network such as:
        \begin{itemize}
            \item Public-key (asymetrical) cryptography
            \item Cryptographic hashes (e.g. SHA-256 Bitcoin)
            \item \ldots (probably more)
        \end{itemize}

        PoW / PoS

        - How does it work? Decentralization, mempool

        Honest nodes control the network
    \end{comment}



    %%%%%%%%%%%%%%%%%%%%% Custom title %%%%%%%%%%%%%%%%%%%%%
    \begin{frame}
        \begin{center}
            \vspace{-4mm}\begin{center}
                             \includegraphics[scale=0.15]{../img/gft.jpg}
            \end{center}

            %\rule{\linewidth}{1.5pt} \\[3mm]
            \vspace{1cm}
            {\huge \bfseries \textcolor{MidnightBlue!100!bg}{ An introduction to\\[3mm] \textcolor{DarkBlue}{Blockchain} technology }} \\[3mm]
            %\rule{\linewidth}{1.5pt} \\[3mm]

            \vspace{1 cm}
            Gerard Bosch (gerard.bosch@gmail.com)

            \vspace{0.8cm}\today
        \end{center}
    \end{frame}
    %%%%%%%%%%%%%%%%%%%%%% End title %%%%%%%%%%%%%%%%%%%%%%%



    %\maketitle
    \frame[shrink]{\frametitle{Outline}\tableofcontents[hideallsubsections]}


    \section{Preliminary concepts}
    %--------------------------------Frame---------------------------------
    \begin{frame}
        \frametitle{What is a Blockchain?}

        \begin{overlayarea}{\textwidth}{\textheight}

            \vspace{3ex}

            \begin{itemize}
                \itemsep=1.35ex
                \item A \alert{distributed cryptographic ledger} shared amongst all nodes participating in a network, over which every transaction is recorded.
                \item<5-> Blockchain serves as the \alert{underlying} technology of several \st{cryptocurrencies} such as Bitcoin.
                \item<7-> The concept and its implementation was created in 2008/2009 and announced in a 9 pages paper writen by Satoshi Nakamoto.
            \end{itemize}

            \only<2> {
            \begin{exampleblock}{Ledger}
                The \st{foundation of accounting}, are as ancient as writing and money.
            \end{exampleblock}
            }

            \only<3> {
            \begin{exampleblock}{Cryptographic}
                The procedures and protocols to \st{append} new data to the ledger implies the use of cryptographic techniques.
            \end{exampleblock}
            }

            \only<4> {
            \begin{exampleblock}{Distributed}
                Not a single entity is the owner of the data, but it is \st{replicated} in every participant of the network.
            \end{exampleblock}
            }

            \only<6> {
            \begin{exampleblock}{Bitcoin}
                was the first and most popular \st{\eh{peer-to-peer} value exchange} network.
            \end{exampleblock}
            }

            \only<8> {
            \begin{exampleblock}{Satoshi Nakamoto}
                is a pseudonym of an anonymous individual or group that developed the idea of Blockchain and Bitcoin.
            \end{exampleblock}
            }

        \end{overlayarea}
    \end{frame}
    %--------------------------------End-----------------------------------


    %--------------------------------Frame---------------------------------
    \begin{frame}
        \frametitle{What is a Blockchain?}

        \begin{overlayarea}{\textwidth}{\textheight}

            \vspace{4ex}

            \only<1-> {
            \centering {\LARGE Now we know, but\ldots how does it look like?}
            }

            \only<2-> {
            \vspace{4ex}
            \includegraphics[scale=0.26]{../img/block-chain.png}
            }

        \end{overlayarea}
    \end{frame}
    %--------------------------------End-----------------------------------


    %--------------------------------Frame---------------------------------
    \begin{frame}
        \frametitle{What is a Blockchain?}

        \begin{overlayarea}{\textwidth}{\textheight}

            \vspace{4ex}

            \only<1-> {
            \centering {\huge Cool! But why?}
            }

            \only<2-> {
            \vspace{4ex}
            \begin{itemize}
                \item \alert{Supress} the trusted third-party (i.e. Financial institutions and Banks).
                \item \alert{Empower} people.
                \item Almost \alert{immediate} transactions.
                \item
                % TODO
            \end{itemize}

            }

        \end{overlayarea}
    \end{frame}
    %--------------------------------End-----------------------------------


    %--------------------------------Frame---------------------------------
    \begin{frame}
        \frametitle{A bit more background}

        \begin{itemize}
            \item Since Bitcoin appearance in 2009, several other \st{cryptocurrencies} emerged.
            \pause
            \item Currently most of them are based in some kind of Blockchain.
            \pause
            \item Blockchain provides a reliable infrastructure that provides \st{at least} 2 out of the 3 properties of \st{CIA triad}: \alert{integrity} and \alert{availability}.
        \end{itemize}

        \begin{exampleblock}{Integrity}
            <4->
            The use of asymetric cryptography guarantees the integrity of data.
        \end{exampleblock}

        \begin{exampleblock}{Availability}
            <5->
            As a decentralized network, there is no single point of failure.
        \end{exampleblock}

        \begin{block}{Confidentiality}
            <6->
            Some implementations seems that could provide it as well (ZCash?).
        \end{block}

    \end{frame}
    %--------------------------------End-----------------------------------

    \section{How does it work?}
    \subsection{Consensus}
    %--------------------------------Frame---------------------------------
    \begin{frame}
        \frametitle{How does it work?}

        \begin{overlayarea}{\textwidth}{0.7\textheight}

            \only<1-> {
            \begin{center}
                \Huge \ts{``It is all about consensus''}
            \end{center}
            }

            \only<2-6> {
            \begin{itemize}
                \itemsep=1ex
                \pause
                \item Blockchain concept is in continuous \st{evolution} and new protocols are continuously created to improve the current flaws.
                \pause
                \item Earliest implementations (which includes Bitcoin and Ethereum) are using a system called \eh{Proof of Work} (\st{PoW}) to \alert{validate} the transactions.
                \pause
                \item \st{Validation} is required in order to append a new block of transactions to the chain; preventing things such as double spend.
                \pause
                \item The process of block validation is known as \st{mining}.
                \pause
                \item Lately a new system called \ts{Proof of Stake} (\st{PoS}) was developed to address PoW drawbacks.
            \end{itemize}
            }

            \only<7-> {
            \begin{itemize}
                \itemsep=1ex
                \item<7-> Nodes are motivated to maintain the network with a \st{reward} coming from transaction fees.
                \item<8-> Hence, \alert{consensus} is achieved though these systems.
            \end{itemize}
            }

        \end{overlayarea}
    \end{frame}
    %--------------------------------End-----------------------------------


    %--------------------------------Frame---------------------------------
    \begin{frame}
        \frametitle{Transaction workflow}

        \begin{enumerate}
            \itemsep=4ex
            \item Clients creates and \alert{signs} transactions (TX) using its private key, then they \st{broadcasts} TX to the network.
            \item Network nodes (miners) receives transactions and stores them in the so called \st{mempool}.
            \item Miners \alert{prioritize} transactions based on fees, \alert{validate} and \alert{put} them in a block.
            \item Once successfully created and \alert{verified} by the network, the block is finally \alert{appended} to the chain.
        \end{enumerate}

    \end{frame}
    %--------------------------------End-----------------------------------


    %--------------------------------Frame---------------------------------
    \begin{frame}
        \frametitle{}

        \begin{center}
            \huge But how does it work under the hood?
        \end{center}

    \end{frame}
    %--------------------------------End-----------------------------------


    \subsection{Proof of Work}
    %--------------------------------Frame---------------------------------
    \begin{frame}
        \frametitle{Proof of Work: The Bitcoin case}
        \begin{overlayarea}{\textwidth}{\textheight}

            \only<1-2> {
            \vspace{2ex}
            \begin{exampleblock}{Block creation (mining)}
                Participants of a Blockchain network put computational \alert{resources} to validate transactions by \alert{solving} the so called \st{cryptographic puzzles}.
            \end{exampleblock}
            }

            \only<2> {
            \begin{itemize}
                \itemsep=1ex
                \item Block validation consist in finding a \st{nonce} (number) for the block that \st{satisfies} a property of the block's hash (a number of leading zeroes).
                \item This is a trial and error procedure (a kind of brute-force).
                \item The first node that find a successful solution \st{announce} it to the network.
                \item The rest of the nodes can \alert{easily verify} that the solution (and hence the block) is valid.
                \item If a node acts \alert{dishonestly}, the rest of nodes will discard the block.
            \end{itemize}
            }

            %    \visible<3> {
            %        \begin{block}{How?}
            %            Taking the solution (nonce) into the block and computing block's hash (SHA-256) must result in a hash with a leading number of zeroes.\\
            %            This is easy to verify for any node.
            %        \end{block}
            %    }

            % Drawbacks
            \only<3> {
            \vspace{2ex}
            \begin{alertblock}{Drawbacks}
                \begin{itemize}
                    \item Huge energy consumption.
                    \item Susceptible to a 51\% attack.
                    \item Democratization of the network (hardware, electricity price,\ldots)
                \end{itemize}
            \end{alertblock}
            }

        \end{overlayarea}
    \end{frame}
    %--------------------------------End-----------------------------------


    \subsection{Proof of Stake}
    %--------------------------------Frame---------------------------------
    \begin{frame}
        \frametitle{Proof of Stake}
        \begin{overlayarea}{\textwidth}{0.85\textheight}

            \only<1-2> {
            Given the aforementioned problems that PoW presents, the new Proof of Stake (PoS) model was developed.

            \begin{exampleblock}{Block creation (forging)}
                Participants of the network \alert{stake} an amount of currency they hold (a kind of deposit) to be able to forge and \alert{send} a block to the network.
            \end{exampleblock}
            }

            \only<2> {
            \begin{itemize}
                \item The next block creator (called forger) will be chosen randomly following certain criteria.
                \item The forger \st{verifies} transactions, \st{forges} a new block and \st{sends} it to the network.
                \item Such as in PoW, new block is added to the chain and forger receives transaction fees (and its stake back).
                \item If the forger acts \alert{dishonestly}, the rest of nodes will discard the block and forger will \alert{lose} the \st{stake}.
            \end{itemize}
            }

            \only<3> {
            \begin{exampleblock}{Pros}
                \begin{itemize}
                    \item A way more \st{energy} efficient: there are no computational resources required.
                    \item More democratization and hence \st{decentralization}.
                    \item \st{Security}: Purchasing more than half of the coins is likely more costly than acquiring 51\% of PoS hashing power.
                \end{itemize}
            \end{exampleblock}

            \begin{alertblock}{}
                Several proposals has been presented, studied and even implemented but PoS faces some \alert{challenges} that must be addressed.
            \end{alertblock}

            }

            \only<4> {
            \vspace{2mm}
            \centering
            {\huge \sl{`` Not so \alert{trivial} ''}}

            \vspace{3ex}
            \begin{columns}[c]
                \column{0.5\textwidth}
                \includegraphics[scale=0.245]{../img/ouroboros-1.png}

                \column{0.5\textwidth}
                \includegraphics[scale=0.255]{../img/ouroboros-2.png}
            \end{columns}
            \vspace{1ex}
            {\footnotesize Ouroboros: A Provably Secure Proof-of-Stake Blockchain Protocol}
            }

        \end{overlayarea}
    \end{frame}
    %--------------------------------End-----------------------------------


    %--------------------------------Frame---------------------------------
    \begin{frame}
        \frametitle{}

    \end{frame}
    %--------------------------------End-----------------------------------


    %--------------------------------Frame---------------------------------
    \begin{frame}
        \frametitle{}

    \end{frame}
    %--------------------------------End-----------------------------------

    \section{Blockchain by generations}
    \subsection{First generation}
    %--------------------------------Frame---------------------------------
    \begin{frame}
        \frametitle{}

    \end{frame}
    %--------------------------------End-----------------------------------


    %--------------------------------Frame---------------------------------
    \begin{frame}
        \frametitle{}

    \end{frame}
    %--------------------------------End-----------------------------------


    %--------------------------------Frame---------------------------------
    \begin{frame}
        \frametitle{}

    \end{frame}
    %--------------------------------End-----------------------------------


    %--------------------------------Frame---------------------------------
    \begin{frame}
        \frametitle{}

    \end{frame}
    %--------------------------------End-----------------------------------

    \section{Cardano: A scientific research driven Blockhain}
    %--------------------------------Frame---------------------------------
    \begin{frame}
        \frametitle{Cardano}

        \centering
        \includegraphics[scale=0.07]{../img/cardano.png}

        \begin{itemize}
            \itemsep=1.35ex
            \item Born in 2015 as an effort to \st{change the way} cryptocurrencies are designed and developed.
            \item Developed together for IOHK company and several universities.
            \item \st{Scientific} research \st{model} and \alert{peer review}.
            \item The Blockchain for ADA cryptocurrency.
            \item Considered a \alert{3rd generation} Blockchain.
            \item Different approach: \alert{How to scale} instead of how many TPS.
            \item Current development roadmap planned at least until 2020.
            \item ADA was launched to trade at October 2017.
        \end{itemize}


    \end{frame}
    %--------------------------------End-----------------------------------


    %--------------------------------Frame---------------------------------
    \begin{frame}
        \frametitle{Cardano}

        \begin{exampleblock}{Key features}
            \begin{itemize}
                \item \st{Proof of Stake} (Ouroboros consensus algorithm)
                \item More sustainable ecosystem.
                \item Scalability
                \item \st{Interoperability} with other Blockchains
                \item Smart contracts
                \item Treasury
                \item Based on \st{epochs} and \st{quorums}
                \item \st{Paralelize} transactions amongst quorums will allow to \alert{scale}.
                \item Reduces network \st{pressure} by using RINA.
            \end{itemize}
        \end{exampleblock}

    \end{frame}
    %--------------------------------End-----------------------------------


    %--------------------------------Frame---------------------------------
    \begin{frame}
        \frametitle{Cardano}
        \begin{overlayarea}{\textwidth}{0.8\textheight}

            \only<1-> {
            Aims to \alert{solve} 3 main problems of current cyrptocurrencies:
            \begin{itemize}
                \item Scalability
                \item Interoperability
                \item Sustainability
            \end{itemize}
            }

            \only<2> {
            \begin{block}{Scalability}

                \begin{tabular}{lcl}
                    \st{PoS} and  \st{parallelization} of epochs &  $\longrightarrow$ &  $\Delta$ TPS {\footnotesize (Transactions per second)} \\
                    Split network in \st{subnets} (RINA)         &  $\longrightarrow$ &  $\nabla$ Bandwidth \\
                    Pruning, compression, partitioning &  $\longrightarrow$ &  $\nabla$ Storage \\

                \end{tabular}

            \end{block}
            }

            \only<3> {
            \begin{block}{Interoperability}
                Allow different cryptocurrencies to \st{talk each other}.

                Allows \st{metadata} into TX $\longrightarrow$ Better integration with banks/govs.

            \end{block}
            }

            \only<4> {
            \begin{block}{Interoperability}

            \end{block}
            }


        \end{overlayarea}
    \end{frame}
    %--------------------------------End-----------------------------------

    %--------------------------------Frame---------------------------------
    \begin{frame}
        \frametitle{}

    \end{frame}
    %--------------------------------End-----------------------------------


    %--------------------------------Frame---------------------------------
    \begin{frame}
        \frametitle{}

    \end{frame}
    %--------------------------------End-----------------------------------


    %--------------------------------Frame---------------------------------
    \begin{frame}
        \frametitle{}

    \end{frame}
    %--------------------------------End-----------------------------------


    %--------------------------------Frame---------------------------------
    \begin{frame}
        \frametitle{}

    \end{frame}
    %--------------------------------End-----------------------------------


    %--------------------------------Frame---------------------------------
    \begin{frame}
        \frametitle{}

    \end{frame}
    %--------------------------------End-----------------------------------


    %--------------------------------Frame---------------------------------
    \begin{frame}
        \frametitle{}

    \end{frame}
    %--------------------------------End-----------------------------------


    %--------------------------------Frame---------------------------------
    \begin{frame}
        \frametitle{}

    \end{frame}
    %--------------------------------End-----------------------------------


    %--------------------------------Frame---------------------------------
    \begin{frame}
        \frametitle{}

    \end{frame}
    %--------------------------------End-----------------------------------


    %--------------------------------Frame---------------------------------
    \begin{frame}
        \frametitle{}

    \end{frame}
    %--------------------------------End-----------------------------------


    %--------------------------------Frame---------------------------------
    \begin{frame}
        \frametitle{}

    \end{frame}
    %--------------------------------End-----------------------------------


    %--------------------------------Frame---------------------------------
    \begin{frame}
        \frametitle{}

    \end{frame}
    %--------------------------------End-----------------------------------


    %--------------------------------Frame---------------------------------
    \begin{frame}
        \frametitle{}

    \end{frame}
    %--------------------------------End-----------------------------------


    %--------------------------------Frame---------------------------------
    \begin{frame}
        \frametitle{}

    \end{frame}
    %--------------------------------End-----------------------------------


    %--------------------------------Frame---------------------------------
    \begin{frame}
        \frametitle{}

    \end{frame}
    %--------------------------------End-----------------------------------


    %--------------------------------Frame---------------------------------
    \begin{frame}
        \frametitle{}

    \end{frame}
    %--------------------------------End-----------------------------------


    %--------------------------------Frame---------------------------------
    \begin{frame}
        \frametitle{}

    \end{frame}
    %--------------------------------End-----------------------------------


    %--------------------------------Frame---------------------------------
    \begin{frame}
        \frametitle{}

    \end{frame}
    %--------------------------------End-----------------------------------


    %--------------------------------Frame---------------------------------
    \begin{frame}
        \frametitle{}

    \end{frame}
    %--------------------------------End-----------------------------------


    %--------------------------------Frame---------------------------------
    \begin{frame}
        \frametitle{}

    \end{frame}
    %--------------------------------End-----------------------------------


    %--------------------------------Frame---------------------------------
    \begin{frame}
        \frametitle{}

    \end{frame}
    %--------------------------------End-----------------------------------






\end{document}


%--------------------------------Frame---------------------------------
\begin{frame}
    \frametitle{}

\end{frame}
%--------------------------------End-----------------------------------